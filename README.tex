% README.tex
% https://github.com/nard-tech/nard-tex-law-package

\documentclass[oneside,10pt,a4paper]{jsarticle}

\usepackage[dvips]{graphicx}
\usepackage[dvips]{graphicx,color}
\usepackage{txfonts} % textsf を正しく表示するため
\usepackage{ascmac} % itembox を使いたいため

\usepackage{article_number}
\usepackage{headers_ja}
\usepackage{replaced_words}

\title{nard-tex-law-package}
\author{Fujita Shu}

\begin{document}
  \maketitle

  \section{article\_number - 条文番号のコマンド}

  \verb|article_number| パッケージを読み込むと、
  条文番号を表示するための\\
  \verb|\LawArticleNumber| コマンドが使用できる。

  \begin{itemize}
    \item \verb|\LawArticleNumber{1}| - \LawArticleNumber{1}
    \item \verb|\LawArticleNumber{2}<1>| - \LawArticleNumber{2}<1>
    \item \verb|\LawArticleNumber{3}<1.1>| - \LawArticleNumber{3}<1.1>
  \end{itemize}

  \newpage

  \section{headers\_ja - 条文のコマンド}

  \verb|headers_ja| パッケージを読み込むと、
  条文の冒頭の番号、内容を表示するためのコマンドが使用できる。

  \subsection{LawArticle}

  \begin{itemize}
    \item \verb|\LawArticle{条文番号}|
      \begin{itembox}[l]{コード}
        {\footnotesize\begin{verbatim}
\LawArticle{1}
\begin{quote}
  これは第1条の内容です。
\end{quote}\end{verbatim}}
      \end{itembox}
      \begin{itembox}[l]{表示内容}
        \LawArticle{1}
        \begin{quote}
          これは第1条の内容です。
        \end{quote}
      \end{itembox}
    \item \verb|\LawArticle{条文番号}<枝番>|
      \begin{itembox}[l]{コード}
        {\footnotesize\begin{verbatim}
\LawArticle{1}<2>
\begin{quote}
  これは第1条の2の内容です。
\end{quote}\end{verbatim}}
      \end{itembox}
      \begin{itembox}[l]{表示内容}
        \LawArticle{1}<2>
        \begin{quote}
          これは第1条の2の内容です。
        \end{quote}
      \end{itembox}
    \item \verb|\LawArticle{条文番号}[内容]|
      \begin{itembox}[l]{コード}
        {\footnotesize\begin{verbatim}
\LawArticle{2}[定義]
\begin{quote}
  これは第2条の内容です。
\end{quote}\end{verbatim}}
      \end{itembox}
      \begin{itembox}[l]{表示内容}
        \LawArticle{2}[定義]
        \begin{quote}
          これは第2条の内容です。
        \end{quote}
      \end{itembox}
    %
    \newpage
    %
    \item \verb|\LawArticle{条文番号}<枝番>[内容]|
      \begin{itembox}[l]{コード}
        {\footnotesize\begin{verbatim}
\LawArticle{2}<2>[定義]
\begin{quote}
  これは第2条の2の内容です。
\end{quote}\end{verbatim}}
      \end{itembox}
      \begin{itembox}[l]{表示内容}
        \LawArticle{2}<2>[定義]
        \begin{quote}
          これは第2条の2の内容です。
        \end{quote}
      \end{itembox}
    %
  \end{itemize}

  \subsection{DeletedLawArticle}

  \begin{itemize}
    \item \verb|\DeletedLawArticle{条文番号}|
      \begin{itembox}[l]{コード}
        {\footnotesize\begin{verbatim}
\DeletedLawArticle{3}
\begin{quote}
  削除
\end{quote}\end{verbatim}}
      \end{itembox}
      \begin{itembox}[l]{表示内容}
        \DeletedLawArticle{3}
        \begin{quote}
          削除
        \end{quote}
      \end{itembox}
    \item \verb|\DeletedLawArticle{条文番号}<枝番>|
      \begin{itembox}[l]{コード}
        {\footnotesize\begin{verbatim}
\DeletedLawArticle{3}<2>
\begin{quote}
  削除
\end{quote}\end{verbatim}}
      \end{itembox}
      \begin{itembox}[l]{表示内容}
        \DeletedLawArticle{3}<2>
        \begin{quote}
          削除
        \end{quote}
      \end{itembox}
    %
    \newpage
    %
    \item \verb|\DeletedLawArticle{条文番号}[内容]|
      \begin{itembox}[l]{コード}
        {\footnotesize\begin{verbatim}
\DeletedLawArticle{4}[定義]
\begin{quote}
  削除
\end{quote}\end{verbatim}}
      \end{itembox}
      \begin{itembox}[l]{表示内容}
        \LawArticle{4}[定義]
        \begin{quote}
          削除
        \end{quote}
      \end{itembox}
    \item \verb|\LawArticle{条文番号}<枝番>[内容]|
      \begin{itembox}[l]{コード}
        {\footnotesize\begin{verbatim}
\LawArticle{4}<2>[定義]
\begin{quote}
  削除
\end{quote}\end{verbatim}}
      \end{itembox}
      \begin{itembox}[l]{表示内容}
        \LawArticle{4}<2>[定義]
        \begin{quote}
          削除
        \end{quote}
      \end{itembox}
    %
  \end{itemize}

  \newpage

  \subsection{LawParagraph}

  \begin{itemize}
    \item \verb|\LawParagraph{項番号}|
      \begin{itembox}[l]{コード}
        {\footnotesize\begin{verbatim}
\LawArticle{5}[定義]
\begin{quote}
  これは第5条の内容です。
  \LawParagraph{2}
  \begin{quote}
    これは第5条第2項の内容です。
  \end{quote}
\end{quote}\end{verbatim}}
      \end{itembox}
      \begin{itembox}[l]{表示内容}
        \LawArticle{5}[定義]
        \begin{quote}
          これは第5条の内容です。
          \LawParagraph{2}
          \begin{quote}
            これは第5条第2項の内容です。
          \end{quote}
        \end{quote}
      \end{itembox}
    %
  \end{itemize}

  \subsection{LawSubsection}

  \begin{itemize}
    \item \verb|\LawSubsection{款番号}{内容}|
      \begin{itembox}[l]{コード}
        {\footnotesize\begin{verbatim}
\LawSubsection{6}{款の内容}
\LawArticle{100}[総則]
\begin{quote}
  これは第100条の内容です。
\end{quote}\end{verbatim}}
      \end{itembox}
      \begin{itembox}[l]{表示内容}
        \LawSubsection{6}{款の内容}
        \LawArticle{100}[総則]
        \begin{quote}
          これは第100条の内容です。
        \end{quote}
      \end{itembox}
    %
  \end{itemize}

  \newpage

  \section{words - 言葉の置換のコマンド}

  \verb|words| パッケージを読み込むと、
  旧仮名遣いなどの表記のゆれを解消するためのコマンドが使用できる。

  \begin{itemize}
    \item \verb|\Atatte| - \Atatte{\footnotesize (← あたつて)}
    \item \verb|\AtatteKanji| - \AtatteKanji{\footnotesize (← 当たつて)}
    \item \verb|\Assen| - \Assen{\footnotesize (← あつ旋)}
    \item \verb|\Atta| - \Atta{\footnotesize (← あつた)}
    \item \verb|\Atte| - \Atte{\footnotesize (← あつて)}
    \item \verb|\Ayamatta| - \Ayamatta{\footnotesize (← 誤つた)}
    \item \verb|\Ayamatte| - \Ayamatte{\footnotesize (← 誤つて)}
    \item \verb|\Itatta| - \Itatta{\footnotesize (← 至つた)}
    \item \verb|\Itatte| - \Itatte{\footnotesize (← 至つて)}
    \item \verb|\Itsuwatte| - \Itsuwatte{\footnotesize (← 偽つて)}
    \item \verb|\Ukai| - \Ukai{\footnotesize (← う回)}
    \item \verb|\Ushinatta| - \Ushinatta{\footnotesize (← 失つた)}
    \item \verb|\Uchikitte| - \Uchikitte{\footnotesize (← 打ち切つて)}
    \item \verb|\Ounatsu| - \Ounatsu{\footnotesize (← 押なつ)}
    \item \verb|\Okonatta| - \Okonatta{\footnotesize (← 行つた)}
    \item \verb|\Okonatte| - \Okonatte{\footnotesize (← 行つて)}
    \item \verb|\Owatta| - \Owatta{\footnotesize (← 終つた)}
    \item \verb|\KakkoWord| - \KakkoWord{\footnotesize (← かつこ)}
    \item \verb|\Kagitte| - \Kagitte{\footnotesize (← 限つて)}
    \item \verb|\Kawatte| - \Kawatte{\footnotesize (← 代わつて)}
    \item \verb|\Kuwawatta| - \Kuwawatta{\footnotesize (← 加わつた)}
    \item \verb|\Gonatsu| - \Gonatsu{\footnotesize (← 誤なつ)}
    \item \verb|\Kotonatte| - \Kotonatte{\footnotesize (← 異なつて)}
    \item \verb|\Shitagatte| - \Shitagatte{\footnotesize (← したがつて)}
    \item \verb|\ShitagatteKanji| - \ShitagatteKanji{\footnotesize (← 従つて)}
    \item \verb|\Shitamawatte| - \Shitamawatte{\footnotesize (← 下回つて)}
    \item \verb|\Shitta| - \Shitta{\footnotesize (← 知つた)}
    \item \verb|\Shinakatta| - \Shinakatta{\footnotesize (← しなかつた)}
    \item \verb|\Shiharatta| - \Shiharatta{\footnotesize (← 支払つた)}
    \item \verb|\Shiharatte| - \Shiharatte{\footnotesize (← 支払つて)}
    \item \verb|\Dekinakatta| - \Dekinakatta{\footnotesize (← できなかつた)}
    \item \verb|\Tokusoku| - \Tokusoku{\footnotesize (← とくそく)}
    \item \verb|\Totte| - \Totte{\footnotesize (← とつて)}
    \item \verb|\Totonotte| - \Totonotte{\footnotesize (← 整つて)}
    \item \verb|\Tomonatte| - \Tomonatte{\footnotesize (← 伴つて)}
    \item \verb|\Toriatsukatta| - \Toriatsukatta{\footnotesize (← 取り扱つた)}
    \item \verb|\Toriatsukatte| - \Toriatsukatte{\footnotesize (← 取り扱つて)}
    \item \verb|\Nakatta| - \Nakatta{\footnotesize (← なかつた)}
    \item \verb|\Nakunatta| - \Nakunatta{\footnotesize (← なくなつた)}
    \item \verb|\Nakunatte| - \Nakunatte{\footnotesize (← なくなつて)}
    \item \verb|\Natta| - \Natta{\footnotesize (← なつた)}
    \item \verb|\Natte| - \Natte{\footnotesize (← なつて)}
    \item \verb|\Hakatta| - \Hakatta{\footnotesize (← 図つた)}
    \item \verb|\Hasu| - \Hasu{\footnotesize (← は数)}
    \item \verb|\Hatte| - \Hatte{\footnotesize (← はつて)}
    \item \verb|\Motte| - \Motte{\footnotesize (← もつて)}
    \item \verb|\Yotte| - \Yotte{\footnotesize (← よつて)}
    \item \verb|\Matagatte| - \Matagatte{\footnotesize (← またがつて)}
    \item \verb|\Moyori| - \Moyori{\footnotesize (← もより)}
    \item \verb|\Watatte| - \Watatte{\footnotesize (← わたつて)}
    \item \verb|\CdRom| - \CdRom{\footnotesize (← シー・ディー・ロム)}
  \end{itemize}
\end{document}

% README.tex
% https://github.com/nard-tech/nard-tex-law-package

\documentclass[oneside,10pt,a4paper]{jsarticle}

\usepackage[dvips]{graphicx}
\usepackage[dvips]{graphicx,color}
\usepackage{txfonts} % textsf を正しく表示するため
\usepackage{ascmac} % itembox を使いたいため
\usepackage{longtable}

\usepackage{article_number}
\usepackage{headers_ja}
\usepackage{replaced_words}
\usepackage{law_utils}

\title{nard-tex-law-package}
\author{Fujita Shu}

\begin{document}
  \maketitle

  \section{article\_number - 条文番号のコマンド}

  \verb|article_number| パッケージを読み込むと、
  条文番号を表示するための\\
  \verb|\LawArticleNumber| コマンドが使用できる。

  \begin{itemize}
    \item \verb|\LawArticleNumber{1}| - \LawArticleNumber{1}
    \item \verb|\LawArticleNumber{2}<1>| - \LawArticleNumber{2}<1>
    \item \verb|\LawArticleNumber{3}<1.1>| - \LawArticleNumber{3}<1.1>
  \end{itemize}

  \newpage

  \section{headers\_ja - 条文のコマンド}

  \verb|headers_ja| パッケージを読み込むと、
  条文の冒頭の番号、内容を表示するためのコマンドが使用できる。

  \subsection{LawArticle}

  \begin{itemize}
    \item \verb|\LawArticle{条文番号}|
      \begin{itembox}[l]{コード}
        {\footnotesize\begin{verbatim}
\LawArticle{1}
\begin{quote}
  これは第1条の内容です。
\end{quote}\end{verbatim}}
      \end{itembox}
      \begin{itembox}[l]{表示内容}
        \LawArticle{1}
        \begin{quote}
          これは第1条の内容です。
        \end{quote}
      \end{itembox}
    \item \verb|\LawArticle{条文番号}<枝番>|
      \begin{itembox}[l]{コード}
        {\footnotesize\begin{verbatim}
\LawArticle{1}<2>
\begin{quote}
  これは第1条の2の内容です。
\end{quote}\end{verbatim}}
      \end{itembox}
      \begin{itembox}[l]{表示内容}
        \LawArticle{1}<2>
        \begin{quote}
          これは第1条の2の内容です。
        \end{quote}
      \end{itembox}
    \item \verb|\LawArticle{条文番号}[内容]|
      \begin{itembox}[l]{コード}
        {\footnotesize\begin{verbatim}
\LawArticle{2}[定義]
\begin{quote}
  これは第2条の内容です。
\end{quote}\end{verbatim}}
      \end{itembox}
      \begin{itembox}[l]{表示内容}
        \LawArticle{2}[定義]
        \begin{quote}
          これは第2条の内容です。
        \end{quote}
      \end{itembox}
    %
    \newpage
    %
    \item \verb|\LawArticle{条文番号}<枝番>[内容]|
      \begin{itembox}[l]{コード}
        {\footnotesize\begin{verbatim}
\LawArticle{2}<2>[定義]
\begin{quote}
  これは第2条の2の内容です。
\end{quote}\end{verbatim}}
      \end{itembox}
      \begin{itembox}[l]{表示内容}
        \LawArticle{2}<2>[定義]
        \begin{quote}
          これは第2条の2の内容です。
        \end{quote}
      \end{itembox}
    %
  \end{itemize}

  \subsection{DeletedLawArticle}

  \begin{itemize}
    \item \verb|\DeletedLawArticle{条文番号}|
      \begin{itembox}[l]{コード}
        {\footnotesize\begin{verbatim}
\DeletedLawArticle{3}
\begin{quote}
  削除
\end{quote}\end{verbatim}}
      \end{itembox}
      \begin{itembox}[l]{表示内容}
        \DeletedLawArticle{3}
        \begin{quote}
          削除
        \end{quote}
      \end{itembox}
    \item \verb|\DeletedLawArticle{条文番号}<枝番>|
      \begin{itembox}[l]{コード}
        {\footnotesize\begin{verbatim}
\DeletedLawArticle{3}<2>
\begin{quote}
  削除
\end{quote}\end{verbatim}}
      \end{itembox}
      \begin{itembox}[l]{表示内容}
        \DeletedLawArticle{3}<2>
        \begin{quote}
          削除
        \end{quote}
      \end{itembox}
    %
    \newpage
    %
    \item \verb|\DeletedLawArticle{条文番号}[内容]|
      \begin{itembox}[l]{コード}
        {\footnotesize\begin{verbatim}
\DeletedLawArticle{4}[定義]
\begin{quote}
  削除
\end{quote}\end{verbatim}}
      \end{itembox}
      \begin{itembox}[l]{表示内容}
        \LawArticle{4}[定義]
        \begin{quote}
          削除
        \end{quote}
      \end{itembox}
    \item \verb|\LawArticle{条文番号}<枝番>[内容]|
      \begin{itembox}[l]{コード}
        {\footnotesize\begin{verbatim}
\LawArticle{4}<2>[定義]
\begin{quote}
  削除
\end{quote}\end{verbatim}}
      \end{itembox}
      \begin{itembox}[l]{表示内容}
        \LawArticle{4}<2>[定義]
        \begin{quote}
          削除
        \end{quote}
      \end{itembox}
    %
  \end{itemize}

  \newpage

  \subsection{LawParagraph}

  \begin{itemize}
    \item \verb|\LawParagraph{項番号}|
      \begin{itembox}[l]{コード}
        {\footnotesize\begin{verbatim}
\LawArticle{5}[定義]
\begin{quote}
  これは第5条の内容です。
  \LawParagraph{2}
  \begin{quote}
    これは第5条第2項の内容です。
  \end{quote}
\end{quote}\end{verbatim}}
      \end{itembox}
      \begin{itembox}[l]{表示内容}
        \LawArticle{5}[定義]
        \begin{quote}
          これは第5条の内容です。
          \LawParagraph{2}
          \begin{quote}
            これは第5条第2項の内容です。
          \end{quote}
        \end{quote}
      \end{itembox}
    %
  \end{itemize}

  \subsection{LawSubsection}

  \begin{itemize}
    \item \verb|\LawSubsection{款番号}{内容}|
      \begin{itembox}[l]{コード}
        {\footnotesize\begin{verbatim}
\LawSubsection{6}{款の内容}
\LawArticle{100}[総則]
\begin{quote}
  これは第100条の内容です。
\end{quote}\end{verbatim}}
      \end{itembox}
      \begin{itembox}[l]{表示内容}
        \LawSubsection{6}{款の内容}
        \LawArticle{100}[総則]
        \begin{quote}
          これは第100条の内容です。
        \end{quote}
      \end{itembox}
    %
  \end{itemize}

  \newpage

  \section{words - 言葉の置換のコマンド}

  \verb|words| パッケージを読み込むと、
  旧仮名遣いなどの表記のゆれを解消するためのコマンドが使用できる。

  \begin{longtable}{lll}
    \verb|\Atatte| & \Atatte & {\footnotesize (← あたつて)} \\
    \verb|\AtatteKanji| & \AtatteKanji & {\footnotesize (← 当たつて)} \\
    \verb|\Assen| & \Assen & {\footnotesize (← あつ旋)} \\
    \verb|\Atta| & \Atta & {\footnotesize (← あつた)} \\
    \verb|\Atte| & \Atte & {\footnotesize (← あつて)} \\
    \verb|\Ayamatta| & \Ayamatta & {\footnotesize (← 誤つた)} \\
    \verb|\Ayamatte| & \Ayamatte & {\footnotesize (← 誤つて)} \\
    \verb|\Itatta| & \Itatta & {\footnotesize (← 至つた)} \\
    \verb|\Itatte| & \Itatte & {\footnotesize (← 至つて)} \\
    \verb|\Itsuwatte| & \Itsuwatte & {\footnotesize (← 偽つて)} \\
    \verb|\Ukai| & \Ukai & {\footnotesize (← う回)} \\
    \verb|\Ushinatta| & \Ushinatta & {\footnotesize (← 失つた)} \\
    \verb|\Uchikitte| & \Uchikitte & {\footnotesize (← 打ち切つて)} \\
    \verb|\Ounatsu| & \Ounatsu & {\footnotesize (← 押なつ)} \\
    \verb|\Okonatta| & \Okonatta & {\footnotesize (← 行つた)} \\
    \verb|\Okonatte| & \Okonatte & {\footnotesize (← 行つて)} \\
    \verb|\Owatta| & \Owatta & {\footnotesize (← 終つた)} \\
    \verb|\KakkoWord| & \KakkoWord & {\footnotesize (← かつこ)} \\
    \verb|\Kagitte| & \Kagitte & {\footnotesize (← 限つて)} \\
    \verb|\Kawatte| & \Kawatte & {\footnotesize (← 代わつて)} \\
    \verb|\Kuwawatta| & \Kuwawatta & {\footnotesize (← 加わつた)} \\
    \verb|\Gonatsu| & \Gonatsu & {\footnotesize (← 誤なつ)} \\
    \verb|\Kotonatte| & \Kotonatte & {\footnotesize (← 異なつて)} \\
    \verb|\Shitagatte| & \Shitagatte & {\footnotesize (← したがつて)} \\
    \verb|\ShitagatteKanji| & \ShitagatteKanji & {\footnotesize (← 従つて)} \\
    \verb|\Shitamawatte| & \Shitamawatte & {\footnotesize (← 下回つて)} \\
    \verb|\Shitta| & \Shitta & {\footnotesize (← 知つた)} \\
    \verb|\Shinakatta| & \Shinakatta & {\footnotesize (← しなかつた)} \\
    \verb|\Shiharatta| & \Shiharatta & {\footnotesize (← 支払つた)} \\
    \verb|\Shiharatte| & \Shiharatte & {\footnotesize (← 支払つて)} \\
    \verb|\Dekinakatta| & \Dekinakatta & {\footnotesize (← できなかつた)} \\
    \verb|\Tokusoku| & \Tokusoku & {\footnotesize (← とくそく)} \\
    \verb|\Totte| & \Totte & {\footnotesize (← とつて)} \\
    \verb|\Totonotte| & \Totonotte & {\footnotesize (← 整つて)} \\
    \verb|\Tomonatte| & \Tomonatte & {\footnotesize (← 伴つて)} \\
    \verb|\Toriatsukatta| & \Toriatsukatta & {\footnotesize (← 取り扱つた)} \\
    \verb|\Toriatsukatte| & \Toriatsukatte & {\footnotesize (← 取り扱つて)} \\
    \verb|\Nakatta| & \Nakatta & {\footnotesize (← なかつた)} \\
    \verb|\Nakunatta| & \Nakunatta & {\footnotesize (← なくなつた)} \\
    \verb|\Nakunatte| & \Nakunatte & {\footnotesize (← なくなつて)} \\
    \verb|\Natta| & \Natta & {\footnotesize (← なつた)} \\
    \verb|\Natte| & \Natte & {\footnotesize (← なつて)} \\
    \verb|\Hakatta| & \Hakatta & {\footnotesize (← 図つた)} \\
    \verb|\Hasu| & \Hasu & {\footnotesize (← は数)} \\
    \verb|\Hatte| & \Hatte & {\footnotesize (← はつて)} \\
    \verb|\Motte| & \Motte & {\footnotesize (← もつて)} \\
    \verb|\Yotte| & \Yotte & {\footnotesize (← よつて)} \\
    \verb|\Matagatte| & \Matagatte & {\footnotesize (← またがつて)} \\
    \verb|\Moyori| & \Moyori & {\footnotesize (← もより)} \\
    \verb|\Watatte| & \Watatte & {\footnotesize (← わたつて)} \\
    \verb|\CdRom| & \CdRom & {\footnotesize (← シー・ディー・ロム)}
  \end{longtable}

 % TODO: law_utils.sty の README を書く
\end{document}

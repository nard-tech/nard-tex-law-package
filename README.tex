% README.tex
% https://github.com/nard-tech/nard-tex-law-package

\documentclass[oneside,10pt,a4paper]{jsarticle}

\usepackage[dvips]{graphicx}
\usepackage[dvips]{graphicx,color}
\usepackage{txfonts} % textsf を正しく表示するため
\usepackage{ascmac} % itembox を使いたいため

\usepackage{article_number}
\usepackage{headers_ja}

\title{nard-tex-law-package}
\author{Fujita Shu}

\begin{document}
  \maketitle

  \section{article\_number - 条文番号のコマンド}

  \verb|article_number| パッケージを読み込むと、
  条文番号を表示するための\\
  \verb|\LawArticleNumber| コマンドが使用できる。

  \begin{itemize}
    \item \verb|\LawArticleNumber{1}| - \LawArticleNumber{1}
    \item \verb|\LawArticleNumber{2}<1>| - \LawArticleNumber{2}<1>
    \item \verb|\LawArticleNumber{3}<1.1>| - \LawArticleNumber{3}<1.1>
  \end{itemize}

  \newpage

  \section{headers\_ja - 条文のコマンド}

  \verb|headers_ja| パッケージを読み込むと、
  条文の冒頭の番号、内容を表示するためのコマンドが使用できる。

  \subsection{LawArticle}

  \begin{itemize}
    \item \verb|\LawArticle{条文番号}|
      \begin{itembox}[l]{コード}
        {\footnotesize\begin{verbatim}
\LawArticle{1}
\begin{quote}
  これは第1条の内容です。
\end{quote}\end{verbatim}}
      \end{itembox}
      \begin{itembox}[l]{表示内容}
        \LawArticle{1}
        \begin{quote}
          これは第1条の内容です。
        \end{quote}
      \end{itembox}
    \item \verb|\LawArticle{条文番号}<枝番>|
      \begin{itembox}[l]{コード}
        {\footnotesize\begin{verbatim}
\LawArticle{1}<2>
\begin{quote}
  これは第1条の2の内容です。
\end{quote}\end{verbatim}}
      \end{itembox}
      \begin{itembox}[l]{表示内容}
        \LawArticle{1}<2>
        \begin{quote}
          これは第1条の2の内容です。
        \end{quote}
      \end{itembox}
    \item \verb|\LawArticle{条文番号}[内容]|
      \begin{itembox}[l]{コード}
        {\footnotesize\begin{verbatim}
\LawArticle{2}[定義]
\begin{quote}
  これは第2条の内容です。
\end{quote}\end{verbatim}}
      \end{itembox}
      \begin{itembox}[l]{表示内容}
        \LawArticle{2}[定義]
        \begin{quote}
          これは第2条の内容です。
        \end{quote}
      \end{itembox}
    %
    \newpage
    %
    \item \verb|\LawArticle{条文番号}<枝番>[内容]|
      \begin{itembox}[l]{コード}
        {\footnotesize\begin{verbatim}
\LawArticle{2}<2>[定義]
\begin{quote}
  これは第2条の2の内容です。
\end{quote}\end{verbatim}}
      \end{itembox}
      \begin{itembox}[l]{表示内容}
        \LawArticle{2}<2>[定義]
        \begin{quote}
          これは第2条の2の内容です。
        \end{quote}
      \end{itembox}
    %
  \end{itemize}

  \subsection{DeletedLawArticle}

  \begin{itemize}
    \item \verb|\DeletedLawArticle{条文番号}|
      \begin{itembox}[l]{コード}
        {\footnotesize\begin{verbatim}
\DeletedLawArticle{3}
\begin{quote}
  削除
\end{quote}\end{verbatim}}
      \end{itembox}
      \begin{itembox}[l]{表示内容}
        \DeletedLawArticle{3}
        \begin{quote}
          削除
        \end{quote}
      \end{itembox}
    \item \verb|\DeletedLawArticle{条文番号}<枝番>|
      \begin{itembox}[l]{コード}
        {\footnotesize\begin{verbatim}
\DeletedLawArticle{3}<2>
\begin{quote}
  削除
\end{quote}\end{verbatim}}
      \end{itembox}
      \begin{itembox}[l]{表示内容}
        \DeletedLawArticle{3}<2>
        \begin{quote}
          削除
        \end{quote}
      \end{itembox}
    %
    \newpage
    %
    \item \verb|\DeletedLawArticle{条文番号}[内容]|
      \begin{itembox}[l]{コード}
        {\footnotesize\begin{verbatim}
\DeletedLawArticle{4}[定義]
\begin{quote}
  削除
\end{quote}\end{verbatim}}
      \end{itembox}
      \begin{itembox}[l]{表示内容}
        \LawArticle{4}[定義]
        \begin{quote}
          削除
        \end{quote}
      \end{itembox}
    \item \verb|\LawArticle{条文番号}<枝番>[内容]|
      \begin{itembox}[l]{コード}
        {\footnotesize\begin{verbatim}
\LawArticle{4}<2>[定義]
\begin{quote}
  削除
\end{quote}\end{verbatim}}
      \end{itembox}
      \begin{itembox}[l]{表示内容}
        \LawArticle{4}<2>[定義]
        \begin{quote}
          削除
        \end{quote}
      \end{itembox}
    %
  \end{itemize}

  \newpage

  \subsection{LawParagraph}

  \begin{itemize}
    \item \verb|\LawParagraph{項番号}|
      \begin{itembox}[l]{コード}
        {\footnotesize\begin{verbatim}
\LawArticle{5}[定義]
\begin{quote}
  これは第5条の内容です。
  \LawParagraph{2}
  \begin{quote}
    これは第5条第2項の内容です。
  \end{quote}
\end{quote}\end{verbatim}}
      \end{itembox}
      \begin{itembox}[l]{表示内容}
        \LawArticle{5}[定義]
        \begin{quote}
          これは第5条の内容です。
          \LawParagraph{2}
          \begin{quote}
            これは第5条第2項の内容です。
          \end{quote}
        \end{quote}
      \end{itembox}
    %
  \end{itemize}

  \subsection{LawSubsection}

  \begin{itemize}
    \item \verb|\LawSubsection{款番号}{内容}|
      \begin{itembox}[l]{コード}
        {\footnotesize\begin{verbatim}
\LawSubsection{6}{款の内容}
\LawArticle{100}[総則]
\begin{quote}
  これは第100条の内容です。
\end{quote}\end{verbatim}}
      \end{itembox}
      \begin{itembox}[l]{表示内容}
        \LawSubsection{6}{款の内容}
        \LawArticle{100}[総則]
        \begin{quote}
          これは第100条の内容です。
        \end{quote}
      \end{itembox}
    %
  \end{itemize}
\end{document}
